\documentclass[10pt,a4paper]{article}

\input{AEDmacros}
\usepackage{caratula} % Version modificada para usar las macros de algo1 de ~> https://github.com/bcardiff/dc-tex


\titulo{Trabajo Práctico 1}
\subtitulo{Especificación y correctitud}

\fecha{\today}

\materia{Algoritmos y Estructuras de Datos/AED II}
\grupo{Grupo 42}

\integrante{Castello, Ulises}{0224/24}{ulisescastello9@gmail.com}
\integrante{Corral, Nicolás}{0343/23}{n.corral.j@gmail.com}
\integrante{Goldfarb, Bruno}{1164/23}{bgoldfarb2003@gmail.com}
\integrante{Apellido, Nombre4}{004/01}{email4@dominio.com}
% Pongan cuantos integrantes quieran

% Declaramos donde van a estar las figuras
% No es obligatorio, pero suele ser comodo
\graphicspath{{../static/}}

\begin{document}

\maketitle

\section{Especificación}
\subsection{Encontrando el camino}


\begin{questions}

    \question 1) grandesCiudades: A partir de una lista de ciudades, devuelve aquellas que tienen más de 50.000 habitantes. \\ \\
Solución:    \\
    \begin{solution}
        Proc  grandesCiudades (in: ciudades: seq$\langle ciudad \rangle$): seq$\langle ciudad \rangle \lbrace$ \\ 
        Requiere: $\lbrace ciudad.habitantes \geq 0 \rbrace $\\
        $Asegura: \lbrace((\forall i; \mathbb{Z})(0 \leq i  < \mid ciudades \mid)) \rightarrow _L (ciudades[i] \in res) \leftrightarrow (ciudades[i].habitantes \geq 50000) \wedge _L $ \\ $\wedge _L(\forall R1;\langle ciudad \rangle) (R1, \in res)  \rightarrow((R1 \in  ciudades) \wedge (R1.habitantes \geq 50000))\rbrace \\ \rbrace$
    \end{solution}
    \\\\\\
    \question  2) sumaDeHabitantes: Por cuestiones de planificación urbana, las ciudades registran sus habitantes mayores de edad por un lado y menores de edad por el otro. Dadas dos listas de ciudades del mismo largo con los mismos nombres, una con sus habitantes mayores y otra con sus habitantes menores, este procedimiento debe devolver una lista de ciudades con la cantidad total de sus habitantes. \\ 
    \begin{solution}
    solución: \\
        proc sumaDeHabitantes (in: menoresDeCiudades: seq⟨Ciudad⟩, in mayoresDeCiudades: seq⟨Ciudad⟩) : seq⟨Ciudad⟩ $\lbrace$ \\
        Requiere: $\lbrace \mid menores \mid$ = $\mid mayores \mid  \wedge _L mismasCiudades(menores,mayores) \rbrace$ \\
        Asegura: $\lbrace (\forall ciudad; \langle ciudades \rangle)(ciudad \in res) \rightarrow (\exists i,j; \mathbb{Z})(0  \geq (i,j) \geq \mid menores \mid) \wedge _L (ciudades.nombres = \\ = menores [i].nombres = mayores[i].nombres) \wedge (ciudades.habitantes = menores [i].habitantes + mayores[i].habitantes) $
        $\\ \rbrace$
        pred mismasCiudades(T1: ⟨Ciudad⟩, T2: ⟨Ciudad⟩) $ \lbrace$  \\
        $ (\forall i; \mathbb{Z}) (0 \leq i  < \mid T1 \mid)) \rightarrow _L (T1[i] \in  T2) \wedge (\forall j; \mathbb{Z}) (0 \leq j  < \mid T2 \mid)) \rightarrow _L (T2[i] \in  T1) $ 
    \end{solution}
\end{questions}

Lo principal: las fórmulas. Se puede poner en una linea, como $x_i = x_{i-1} + x_{i-2}$, o ponerse más grande:

\begin{equation}
	\sum\limits_{i=0}^{n} i
	\label{eq:1}
\end{equation}

Y se pueden citar ecuaciones con \eqref{f}|: \eqref{eq:1}

Ejemplo de itemizado:

\begin{itemize}
	\item Item 1
	\item Item 2
	\item Item 3
\end{itemize}

Ejemplo de enumerado con menor distancia entre items:

\begin{enumerate} \setlength\itemsep{0cm}
	\item Item 1
	\item Item 2
	\item Item 3
\end{enumerate}

Podemos escribir mucho texto. Mucho texto. Mucho texto. Mucho texto. Mucho texto. Mucho texto. Mucho texto. Mucho texto. Mucho texto. Mucho texto. Mucho texto.

Otro párrafo. Otro párrafo. Otro párrafo. Otro párrafo. Otro párrafo. Otro párrafo. Otro párrafo. Otro párrafo. Otro párrafo. Otro párrafo. Otro párrafo. Otro párrafo. Otro párrafo.

\vspace{0.3cm}

Le agregamos una separación entre párrafos. Le agregamos una separación entre párrafos. Le agregamos una separación entre párrafos. Le agregamos una separación entre párrafos. Le agregamos una separación entre párrafos.

\vspace{0.3cm}

La tabla \ref{tab:ejemplo} es un ejemplo de cómo se hace una tabla.

\begin{table}[h!]
	\centering
	\begin{tabular}{||l c c r||} 
		\hline
		Col1 & Col2 & Col2 & Col3 \\ [0.5ex] 
		\hline\hline
		1 & 6 & 87837 & 787 \\ 
		2 & 7 & 78 & 5415 \\
		3 & 545 & 778 & 7507 \\
		4 & 545 & 18744 & 7560 \\
		5 & 88 & 788 & 6344 \\
		\hline
	\end{tabular}
	\caption{Ejemplo de tabla}
	\label{tab:ejemplo}
\end{table}


La figura \ref{fig:subfigs} es un ejemplo de cómo se agrega una imagen.

\begin{figure}[ht]
	\centering
	\includegraphics[width=0.6\textwidth]{logo_dc.jpg}
	\caption{Ejemplo de figura}
	\label{fig:ejemplo}
\end{figure}

\begin{figure}[ht!]
	\begin{subfigure}{0.5\textwidth}
		\includegraphics[width=0.9\linewidth]{LaTeX-project} 
		\caption{Logo de LaTeX}
		\label{fig:subfig1}
	\end{subfigure}
	\begin{subfigure}{0.5\textwidth}
		\includegraphics[width=0.7\linewidth]{TeX}
		\caption{Logo de TeX}
		\label{fig:subfig2}
	\end{subfigure}
	\caption{Ejemplo para poner dos figuras juntas. Y citarlas por separado a (\subref{fig:subfig1}) y (\subref{fig:subfig2}).}
	% OJO: el caption siempre va antes del label
	\label{fig:subfigs}
\end{figure}



% Para hacer que quede todo en una misma linea, se puede usar minipage
%\begin{minipage}[t]{\textwidth}
	\begin{lstlisting}[caption={Ejemplo de código (usando los estilos de la cátedra, ver las macros para más detalles)},label=code:for]
res := 0;
i := 0;
while (i < s.size()) do
	res := res + s[i];
	i := i + 1
endwhile
	\end{lstlisting}
%\end{minipage}

Si se pone un label al \verb|lstlisting|, se puede referenciar: Código \ref{code:for}.


\subsection{Macros de la cátedra para especificar}

\begin{proc}{nombre}{\In paramIn : \nat, \Inout paramInout : \TLista{\ent}}{tipoRes}
	%    \modifica{parametro1, parametro2,..}
	\requiere{expresionBooleana1}
	\asegura{expresionBooleana2}
	\aux{auxiliar1}{parametros}{tipoRes}{expresion}
	\pred{pred1}{parametros}{expresion} 
\end{proc}

\aux{auxiliarSuelto}{parametros}{tipoRes}{expresion}
% \paraTodo{variable}{tipo}{expresion}
% \existe{variable}{tipo}{expresion}
% Pueden tener [unalinea] para que no se divida en varias lineas
\pred{predSuelto}{parametros}{\paraTodo[unalinea]{variable}{tipo}{algo \implicaLuego expresion}}
\pred{predSuelto}{parametros}{\existe[unalinea]{variable}{tipo}{algo \yLuego expresion}}



\end{document}
